\chapter{Introduction}

JUnit\index{junit} is a framework ostensibly useful for testing
individual units of a system.  The framework is more generally useful
for testing at different levels, not simply the unit level.  JUnit
has been integrated into build systems such as Jakarta-Ant\index{ant}
and Apache-Maven\index{maven}.  The integration with build tools
allows the tests to be run during the course of every developer
build, ensuring integrity of the system at any time.

Unit tests\index{unit test|see test, unit}\index{test!unit} are
typically created to test a very small, isolated portion of the
system, while 
\emph{system tests}\index{system test|see test, system}\index{test!system}
test larger grained chunks of an application.  System tests may test
complex iteractions between many units simultaneously.  The
concurrency required by many system tests is not directly supported
with the base JUnit framework.  SysUnit\index{sysunit} adds to
the base JUnit framework to support concurrent and distributed
system testing.  By building upon JUnit, SysUnit allows system tests
to be run at the same time, using the same mechanisms, as an
existing base of unit tests.  From the point-of-view of Ant or
Maven, a SysUnit system test is simply yet-another-test.
