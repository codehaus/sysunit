\chapter{System Testing}

Unit tests are by definition tests that verify the operation
of a \emph{unit} of code.  There is no absolute strict definition
of unit but most view a unit to be equivalent to a single 
implementation class.  In some instances, though, a unit may be
composed of multiple classes within a package.  

When testing the interactions of multiple units, you have
graudated to \emph{system testing}, which may verify
more complex behaviour of the system.  Many times system
testing requires concurrently executing several bits of code,
synchronizing their execution, and verifying the results
across the entire environment.

For example, the code being tested may be a distributed
lock manager which maybe control the read and write access
to a particular piece of data from multiple threads. The
test would require at least three threads to robustly
perform the tests.  

\begin{enumerate}
	\item \emph{The server.}  The server thread starts
		the lock manager and allows other threads/clients
		to work with it.
	\item \emph{Client \#1.} One client interacting
		with the lock-manager server.
	\item \emph{Client \#2.} Another client interacting
		with the lock-manager server.
\end{enumerate}

When multiple independent threads are executing a test-case,
many times it's convenient for some to pause until others
have reached a particular state.  For example, the client
threads should not attempt to run their test code until
after the lock-manager server is fully configured and
initialized.

%\begin{ednote}
%More description about how to write good system-tests is 
%needed here.  Examples, etc.
%\end{ednote}
