\chapter{System Testing}

Unit tests are by definition tests that verify the operation
of a \emph{unit} of code.  There is no absolute strict definition
of unit but most view a unit to be equivalent to a single 
implementation class.  In some instances, though, a unit may be
composed of multiple classes within a package.  

When testing the interactions of multiple units, you have
graudated to \emph{system testing}, which may verify
more complex behaviour of the system.  Many times system
testing requires concurrently executing several bits of code,
synchronizing their execution, and verifying the results
across the entire environment.

Throughout this documentation, the imaginary system used for
the examples is an application which is responsible for accepting
orders for books and attempting to fullfill them by generating
pick lists for warehouse workers and freight bills for
the shipping department.

The various units, which may be tested independently include

\begin{itemize}
	\item Order entry
	\item Inventory verification
	\item Pick list production
	\item Freight bill production
\end{itemize}

Each unit is assumed to already be tested and verified
to work in isolation.  The order entry unit successfully
accepts orders and stores them.  The inventory verification
unit can successfully verify the stock levels of a book.
The pick list production unit can successfully produce
a pick list for a given set of books, and the freight bill
unit can successfully produce a freight bill for a given
set of books.

Of interest is whether all of the units may function
in an end-to-end manner.  The system test is responsible
for this verification.
