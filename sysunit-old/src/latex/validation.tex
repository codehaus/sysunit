\chapter{Test Validation\index{validation}}

\section{Introduction}

Code does not a test make.  Only through validating the results
of the code does a test gain true value.  JUnit provides a range
of assertion \index{assertion} methods for validating the state
of a test.  Assertions are a way of validating that the code
being tested produced the expected results.

\section{Validation with \indexclass{AbstractTBean}}

The abstract convenience \indexclass{TBean} implementations of
\indexclass{AbstractTBean} and \\
\indexclass{AbstractSynchronizableTBean} extend the
base JUnit \indexclass{Assert} class, allowing for all
normal JUnit validation methods to be used.

\begin{codelisting}
public class MyFirstTBean
    extends AbstractSynchronizableTBean
\{
    public void run()
        throws Exception
    \{
        String result = doSomething();
        sync( "after-something" );
        assertEquals( "garbonzo",
                      result );
    \}
\}
\end{codelisting}

Additionally, the \class{TBean} class provides for an
\indexmethod{TBean}{assertValid()} method to do validation
after all threads have completed.

\begin{codelisting}
public class MyFirstTBean
    extends AbstractSynchronizableTBean
\{
    public void run()
        throws Exception
    \{
        doSomething();
    \}

    public void assertValid()
        throws Exception
    \{
        assertEquals( this.garbonzo,
                      "garbonzo"
    \}
\}
\end{codelisting}

\section{Validation with \method{threadXXX(...)} methods}

Since \method{threadXXX(...)} methods appear on implements
of \indexclass{SystemTestCase} which itself is a subclass of
JUnit's \indexclass{TestCase}, all \class{Assert} methods
are available.

\begin{codelisting}
public class MyFirstSystemCase
    extends SystemTestCase
\{
    public void threadOne()
    \{
        String result = doSomething();
        sync( "after-something" );
        assertEquals( "garbonzo",
                      result );
    \}
\}
\end{codelisting}

