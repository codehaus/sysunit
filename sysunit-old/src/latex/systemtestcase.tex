\chapter{The \indexclass{SystemTestCase} Class}

\section{Overview}

When using the base JUnit framework, each unit test
is represented by a single method of the form 
\indexmethod{TestCase}{textXXX()} on a subclass of \indexclass{TestCase}.
In this way, each subclass of \class{TestCase} may actually
contain several tests.

Unlike the base JUnit framework, a SysUnit test class creates
only a single test.  A system test is created by subclassing
the \class{SystemTestCase} class.  JUnit's \method{textXXX()}
idiom is \emph{not} used when creating tests, since SysUnit
creates a test-per-class, instead of a test-per-method.

Each portion of the system that interacts within a test
is a \indexclass{TBean}, which may be interpreted as either
a \emph{test bean} or a \emph{thread bean}.  A
\class{TBean} is similar to the standard \class{Runnable}
interface, with a few modifications to assist with testing.
When a system test is executed, all of the test's \class{TBean}
instances are executed in parallel.  The SysUnit framework
handles the spawning of the threads necessarily for 
concurrently executing the code.

There are two different ways to create \class{TBean} instances
for a test:  

\begin{enumerate}
	\item \class{TBean} factory method, using
\indexmethod{SystemTestCase}{tbeanXXX()}.
	\item Thread factory method, using
\indexmethod{SystemTestCase}{threadXXX()}.
\end{enumerate}


\section{\class{TBean} Factory Method}

To configure a unit of code to execute concurrently, a
\emph{TBean factory method} may be used.  It is simply
a method that begins with the prefix \texttt{tbean} and returns
a \indexclass{TBean}.  When the test class is executed, each
\class{TBean} factory method will be called to generate
the set of \class{TBean} instances used for the test.

\begin{codelisting}
public class MySystemTest
  extends SystemTestCase
\{
    public TBean tbeanOrderAcceptor()
    \{
        return new OrderAcceptorTBean();
    \}

    public TBean tbeanInventoryVerifier()
    \{
        return new InventoryVerifierTBean();
    \}
\}
\end{codelisting}

\section{Thread\index{threads} Factory Method}

To aide in creating concurrent tests without the need of
embodying each unit of code within a \class{TBean}, SysUnit
supports the notion of a \emph{thread factory method}.
In addition to the \class{TBean} factory method for creating executable
segments of code, any method that has the prefix of \texttt{thread}
will also be run concurrently.

\begin{note}
The term ``thread'' is used through this manual to denote a
general thread of execution, and not necessarily instances of
\texttt{java.lang.}\indexclass{Thread}, though they may certainly be
implemented using the \class{Thread} class under the covers.
\end{note}

\begin{codelisting}
public class MySystemTest
  extends SystemTestCase
\{
    public void threadOrderAcceptor()
        throws Exception
    \{
         // the logic for running the order acceptor
    \}

    public void threadInventoryVerifier()
        throws Exception
    \{
         // the logic for running the inventory verifier
    \}
\}
\end{codelisting}

\section{Synchronization\index{synchronization}}

Many times, while multiple segments of code are running in independent
threads of execution, it is desirable to synchronize at a point.  In
multi-threaded programming, this is considered a
\emph{barrier}\index{barrier},
which is a point in the code that causes the threads to block and wait
until all threads reach the same point, and then continue
independently once unblocked.

\begin{note}
The term ``synchronization'' implies coordination, not unlike
how secret-agents will synchronize their watches to ensure 
activities happen at the same time.  SysUnit's usage of
``synchronization'' should not be confused with Java's
\texttt{synchronized} keyword, which actually ensures that
multiple things \emph{do not} occur at the same time.
\end{note}

\subsection{\indexclass{Synchronizer}}

The basis for synchronizing across multiple threads (local or
distributed) is the \class{Synchronizer}, which describes a single
\indexmethod{Synchronizer}{sync(...)} method.  

\begin{codelisting}
public interface Synchronizer 
\{
    void sync(String tbeanId,
              String syncPointName)
        throws SynchronizationException, InterruptedException;
\}
\end{codelisting}

The \class{Synchronizer} provides a means for blocking at a named
barrier, or \emph{sync-point}\index{sync-point} until all other threads of execution 
block at a sync-point or have completed.  The \texttt{tbeanId}
parameter identifies the \class{TBean} requesting the synchronization,
while the \texttt{syncPointName} identifies the particular sync-point
to wait for.  The \texttt{tbeanId} is typically derrived from the
\method{tbeanXXX()} and \method{threadXXX()} method names, after
removing the prefix.

The \texttt{syncPointName} is any arbitrarily chosen identifier
for the synchronization rendezvous.  The following example shows
three \class{TBean} instances synchronizing at a sync-point
named ``after-initialization''

\begin{codelisting}
    synchronizer.sync( "FirstTBean",
                       "after-initialization" );
\vskip0.5em
\hrule
\vskip0.5em
    synchronizer.sync( "SecondTBean",
                       "after-initialization" );
\vskip0.5em
\hrule
\vskip0.5em
    synchronizer.sync( "ThirdTBean",
                       "after-initialization" );
\end{codelisting}

The \class{Synchronizer} guarantees that all three threads will
reach the call to \method{sync(...)} and wait until the others
have also reached that point.  All threads will then simultaneously
unblock, allowing execution to continue.
Any number of sync-points\index{sync-point!multiple} may be used through-out a test in order
to allow coordination of multiple phases.

\subsection{\indexclass{SynchronizableTBean},
\indexclass{AbstractSynchronizableTBean}\\
and \indexclass{TBeanSynchronizer}}

The base \class{TBean} interface does not allow for synchronization. The
\class{TBean} sub-interface \class{SynchronizableTBean} adds
\indexmethod{SynchronizableTBean}{setSynchronizer(...)} method which
signals to the SysUnit framework that a specialized \indexclass{TBeanSynchronizer}
should be provided for synchronization within the \class{TBean}. 

\begin{codelisting}
public interface SynchronizableTBean
    extends TBean 
\{
    void setSynchronizer(TBeanSynchronizer synchronizer);
\}
\end{codelisting}

The \indexclass{TBeanSynchronizer} manages the \texttt{tbeanId}
parameter required by \\ \class{Synchronizer.sync(...)}, providing
a simple synchronization interface which requires only a
\texttt{syncPointName} parameter.

\begin{codelisting}
public interface TBeanSynchronizer
\{
    void sync(String syncPointName)
        throws SynchronizationException, InterruptedException;
\}
\end{codelisting}

The \indexclass{AbstractSynchronizableTBean} provides a base
for implementations of \indexclass{SynchronizableTBean} to
manage the \indexclass{TBeanSynchronizer} and provides a convenience
\indexmethod{AbstractSynchronizableTBean}{sync(...)} method.

\begin{codelisting}
public class MyFirstTBean
    extends AbstractSynchronizableTBean
\{
    public void run()
        throws Exception
    \{
        doSomething();
        sync( "after-something" );
        doSomethingElse();
        sync( "after-something-else" );
    \}
\}

\vskip0.5em
\hrule
\vskip0.5em

public class MySecondTBean
    extends AbstractSynchronizableTBean
\{
    public void run()
        throws Exception
    \{
        doDifferentSomething();
        sync( "after-something" );
        doDifferentSomethingElse();
        sync( "after-something-else" );
    \}
\}
\end{codelisting}

In general, \class{TBean} implementations should extend the
\class{AbstractSynchronizableTBean} abstract class to make
writing test code simpler.  Management of the \class{Synchronizer}
and \texttt{tbeanId} will be managed automatically.

\subsection{\class{SystemTestCase}.\indexmethod{SystemTestCase}{sync(...)}}

When using the \method{threadXXX(...)} form for creating implicit
\class{TBean} instances, the \indexclass{SystemTestCase} provides
a simple \indexmethod{SystemTestCase}{sync(...)} identical to the
one provided by \class{AbstractSynchronizableTBean}.  

\begin{codelisting}
public class MySystemTest
    extends SystemTestCase
\{
    public void threadOne()
    \{
        doSomething();
        sync( "after-something" );
    \}

    public void threadTwo()
    \{
        doDifferentSomething();
        sync( "after-something" );
    \}
\}
\end{codelisting}
