\chapter{The \indexclass{SystemTestCase} Class}

When using the base JUnit framework, each unit test
is represented by a single method of the form 
\indexmethod{TestCase}{textXXX()} on a subclass of \indexclass{TestCase}.
In this way, each subclass of \class{TestCase} may actually
contain several tests.

Unlike the base JUnit framework, a SysUnit test class creates
only a single test.  A system test is created by subclassing
the \class{SystemTestCase} class.  JUnit's \method{textXXX()}
idiom is \emph{not} used when creating tests, since SysUnit
creates a test-per-class, instead of a test-per-method.

Each portion of the system that interacts within a test
is a \indexclass{TBean}, which may be interpreted as either
a \emph{test bean} or a \emph{thread bean}.  A
\class{TBean} is similar to the standard \class{Runnable}
interface, with a few modifications to assist with testing.
When a system test is executed, all of the test's \class{TBean}
instances are executed in parallel.  The SysUnit framework
handles the spawning of the threads necessarily for 
concurrently executing the code.

There are two different ways to create \class{TBean} instances
for a test:  

\begin{enumerate}
	\item \class{TBean} factory method, using
\indexmethod{SystemTestCase}{tbeanXXX()}.
	\item Thread factory method, using
\indexmethod{SystemTestCase}{threadXXX()}.
\end{enumerate}


