\chapter{Watchdog Timer}

\section{Overview}

Many times while testing complex systems, threads may become
ill-behaved and never terminate.  The designer of a test may
be able to specify a \emph{reasonable} amount of time
in which a test should successfully complete.  By overrunning
the time limit, it can be assumed that the test has failed
internally somehow.  A \emph{watchdog timer}\index{watchdog timer} 
can keep an eye on the running test and stop the run if
the time limit is exceeded.

\section{Enabling the Watchdog}

In each \class{SystemTestCase},  the method
\indexmethod{SystemTestCase}{getTimeout()}
may be overridden to return a positive integer rerpresenting
the time limit, in milliseconds, placed upon the test.  If
the test does not successfully complete, fail an assertion
or throw an error within the specified timeout a
\indexclass{WatchdogException} will be thrown indicating
an error.  The \class{WatchdogException} will contain
information regarding the threads or \class{TBean} instances
that were still running at the time of expiration.

\begin{figure}
\begin{codelisting}
import org.sysunit.SystemTestCase;

public class TestTheBadServer
    extends SystemTestCase
\{
    public long getTimeout() 
    \{
        return 15 * 1000;
    \}

    public void threadServer() 
    \{
        while ( true ) 
        \{
            Thread.sleep( 1000 );
        \}
    \}
\}
\end{codelisting}
\caption{Example of 15 second watchdog using
\indexmethod{SystemTestCase}{getTimeout()}.}
\end{figure}

\section{Accomodating Different Environments}

While the test designer specifies a reasonable watchdog timeout,
``reasonable'' may differ from one environment to another. In order
to accomodate both higher and lower performance test environments,
the watchdog timers can be adjusted using the
\texttt{org.sysunit.timeout.multiplier}\index{watchdog
multiplier|see timeout multiplier}\index{timeout multiplier} system
property.  This property may be set to any floating-point number to
increase or decrease the actual watchdog timeout used by the test run.


\begin{figure}
\begin{tabular}{c|c|c}
\hline
\textbf{\texttt{getTimeout()}} &
\textbf{org.sysunit.timeout.multiplier} & 
\textbf{actual timeout}\\
\hline 
\emph{(unset)} & \emph{(unset)} & \emph{(none)} \\
\hline
\emph{(unset)} & 2 & \emph{(none)} \\
\hline
15000 & \emph{(unset)} & 15000 \\
\hline
15000 & 2 & 30000 \\
\hline
15000 & 0.25 & 3750 \\
\hline
\end{tabular}
\caption{Examples of \texttt{org.sysunit.timeout.multiplier}.}
\end{figure}


