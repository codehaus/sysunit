\chapter{The Basics}

\section{Overview}

System tests are realized by creating subclasses of the 
\indexclass{SystemTestCase}.  This class is plays a role similar
to that of \indexclass{TestCase} in standard JUnit.  The
\indexclass{SystemTestCase} implements the \indexclass{Test}
interface from JUnit, allowing SysUnit tests to be executed within
a normal test cycle by any normal JUnit test runner.

Compared to standard \indexclass{TestCase}-based tests, which
provide for \emph{test-per-method}, \indexclass{SystemTestCase}-based
tests are larger-grained and use \emph{test-per-class} semantics.
These semantics are required in order to allow a single system 
test to define multiple simultaneous logical threads of execution
which all participate in the test.

\begin{figure}
\begin{codelisting}
import junit.framework.TestCase;

public class MyUnitTest 
    extends TestCase
\{
    public void testSomething() 
        throws Exception 
    \{
        \dots
    \}

    public void testSomethingElse() 
        throws Exception 
    \{
        \dots
    \}
\}
\end{codelisting}
\caption[JUnit \texttt{testXXX()} example]{Example JUnit \indexclass{TestCase}-based test, demonstrating
two related by independent tests named \texttt{testSomething} and
\texttt{testSomethingElse}.}
\end{figure}

\begin{figure}
\begin{codelisting}
import org.sysunit.SystemTestCase;

public class MySystemTest 
    extends SystemTestCase
\{
    public void threadOne() 
        throws Exception 
    \{
        \dots
    \}

    public void threadTwo() 
        throws Exception 
    \{
        \dots
    \}
\}
\end{codelisting}
\caption{Example SysUnit \indexclass{SystemTestCase}-based test, demonstrating
a single test with two logical threads of execution named
\texttt{threadOne} and \texttt{threadTwo}.
\texttt{testSomethingElse}.}
\end{figure}

Whereas JUnit tests define a single thread of execution, SysUnit
allows test writers to define multiple threads of execution and
provides two methods for their creation.  Simple tests can be
created using \emph{thread methods} while more complex
tests may be created using \emph{\indexclass{TBean}s}.

\begin{note}
The term ``thread'' as used in reference to SysUnit's logical
threads of execution should \emph{not} be confused with
the \class{java.lang.Thread} multithreading class provided
by Java.
\end{note}

\section{Thread methods\index{thread method}}

By far, the simplest method for creating multiple logical threads
is by using \emph{thread methods}.  Similar to JUnit's
special handling of \texttt{testXXX()} methods, SysUnit will
inspect test classes for public methods that are named with
the prefix \texttt{thread}.  For example, a test may include
two methods named \texttt{threadClient()} and \texttt{threadServer()}.

\begin{figure}
\begin{codelisting}
import org.sysunit.SystemTestCase;

public class TestTheServer
\{
    public void threadServer()
        throws Exception 
    \{
         \dots
    \}

    public void threadClient()
        throws Exception 
    \{
         \dots
    \}
\}
\end{codelisting}
\caption{Example system test defining two thread methods named
\texttt{threadServer} and \texttt{threadClient}.}
\end{figure}

A test class may define any number of thread methods.  Thread methods
are allowed to throw exceptions and are required to have \texttt{void}
return values.  SysUnit itself is responsible for the execution of
the thread methods.  The thread method itself \emph{should not}
explicitly spawn or start any \class{Thread}s.  The thread method is
simply the body to be run concurrently with other logical threads.
It is somewhat equivalent to \method{run()} method of the
\class{Runnable} class.  When using thread methods, there is no
equivalent to JUnit's \method{setUp()} and \method{tearDown()} 
methods.  For logical threads requiring more complex initialization
and termination, the \indexclass{TBean} method should be used.

\section{\indexclass{TBean} class}

The \indexclass{TBean} class bridges the gap between simplistic
thread methods and the 3-stage standard JUnit testing methodology.
A system test may opt to use \class{TBean} instances instead of
(or in addition to) thread methods.  A \indexclass{TBean} allows
for a \indexmethod{TBean}{run()} method which plays the same role
as a thread method and also provides for \indexmethod{TBean}{setUp()}
and \indexmethod{TBean}{tearDown()} methods for initializing and
terminating each logical thread.
Additionally, \indexclass{TBean} instances may define an
\indexmethod{TBean}{assertValid()} method to do post-test 
validation of its internal state.

\begin{figure}
\begin{codelisting}
import org.sysunit.TBean;

public class ServerTBean 
    implements TBean
\{
    public void setUp()
        throws Exception
    \{
        \dots
    \}

    public void run()
         throws Exception
    \{
        \dots
    \}

    public void tearDown()
        throws Exception
    \{
        \dots
    \}

    public void assertValid()
    \{
        \dots
    \}
\}
\end{codelisting}
\caption{Example \indexclass{TBean} demonstrating initialization,
thread logic, and termination.}
\end{figure}

Like the thread methods, the \indexmethod{TBean}{run()} method of
\class{TBean} is comparable to the \method{run()} method of
\class{Runnable} in that it should not actually spawn any real
threads, but is simply the body to be executed by the SysUnit engine
itself. 

To weave a \class{TBean} into a system test, SysUnit looks for
publc methods named with the prefix \texttt{tbean} that return
\class{TBean}.  These methods should not execute the \class{TBean}
but are merely \emph{factories}\index{factory pattern} to
construct the \class{TBean} instances.

\begin{figure}
\begin{codelisting}
import org.sysunit.SystemTestCase;
import org.sysunit.TBean;

public class TestTheServer 
\{
    public TBean tbeanServer()
        throws Exception 
    \{
        TBean server = \dots
        \dots
        return server;
    \}

    public TBean tbeanClient()
        throws Exception 
    \{
        TBean client = \dots
        \dots
        return client;
    \}
\}
\end{codelisting}
\caption{Example usage of \emph{\class{TBean} factory methods}.}
\end{figure}
