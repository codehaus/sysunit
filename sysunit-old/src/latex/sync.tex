\chapter{Synchronization}

\section{Overview}

For many tests, merely running multiple logical threads
of execution is not sufficient for choreographing a valid
test.  It is valuable to be able to control each thread's
execution in relation to the other threads of the test.
For example, a test that starts a server and multiple clients
may produce erroneous results if the client threads begin their
tasks before the server is fully initialized.

In multi-threaded programming, a construct known as a 
\emph{barrier}\index{barrier}, or checkpoint\index{checkpoint|see barrier}, 
is used to allow threads
to rendezvous at a point in the code and block (or wait)
until other threads have reached the same point.  SysUnit
extends the barrier construct to system tests to allow the
coordination of multiple logical threads.

\begin{note}
The synchronization provided by SysUnit should not be
confused with the \texttt{synchronized} keyword in the
Java language.  The \texttt{synchronized} keyword allows
for Java to ensure that only a single thread is within 
a \emph{critical section}\index{critical section}
of code, while SysUnit \emph{synchronization} implies
coordination, not unlike secret-agents who synchronize their
watches to ensure mayhem ensues in a simultaneous fashion.
\end{note}

\section{Synchronizing Within a Thread Method}

When using \emph{thread methods} within a system
test, the \indexmethod{SystemTestCase}{sync(...)} method
is available to provide rendezvous synchronization.  Named
\emph{sync points}\index{sync point} allow for logical
threads to pause until other threads have reached the same
sync point.

Each thread method may use the \indexmethod{SystemTestCase}{sync(...)}
method to block until all other logical threads have called the
method with the same sync point name.

\begin{figure}
\begin{codelisting}
import org.sysunit.SystemTestCase;

public class TestTheServer
    extends SystemTestCase
\{
    public void threadServer() 
    \{
        startServer();
        sync( "server.started" );
        sync( "clients.done" );
        stopServer();
    \}

    public void threadClientOne()
    \{
        sync( "server.started" );
        runClient( "client one" );
        sync( "clients.done" );
    \}

    public void threadClientTwo()
    \{
        sync( "server.started" );
        runClient( "client two" );
        sync( "clients.done" );
    \}
\}
\end{codelisting}
\caption{Example of \indexmethod{SystemTestCase}{sync(...)} used
within thread methods.}
\label{example.thread.method.sync}
\end{figure}

In figure \vref{example.thread.method.sync} three logical threads
start initially.  Immediately, \texttt{threadClientOne} and
\texttt{threadClientTwo} call \method{sync("server.started")}
which halts their execution.  Only after the \texttt{threadServer}
has started the server does it call \method{sync("server.started")}
which unblocks the clients.  The server thread immediately then
calls \method{sync("clients.done")}, which blocks its execution
until both clients have completed their running and also made
calls to \method{sync("clients.done")}.

\section{Synchronization Within a \indexclass{TBean}}

Using synchronization with a \indexclass{TBean} is a bit trickier,
since \class{TBean} is merely an interface to be implemented.  At
the bare minimum, to synchronize from within a the
\indexmethod{TBean}{run()} method of a \class{TBean}, it is required
that the class implement \indexclass{SynchronizableTBean}.  This
interface provides a single additional method
\indexclass{SynchronizableTBean}{setSynchronizer(...)} which is
used by SysUnit to provide an \indexclass{TBeanSynchronizer}.

The \indexclass{TBeanSynchronizer} provides a
\indexmethod{TBeanSynchronizer}{sync(...)} method that can be
used from within the \method{run()} method in the same manner
as from within a thread method.  See figure
\vref{example.synchronizable.tbean} for an example.


\begin{figure}
\begin{codelisting}
import org.sysunit.SynchronizableTBean;
import org.sysunit.TBeanSynchronizer;

public class ServerTBean 
    implements SynchronizableTBean
\{
    private Server server;
    private TBeanSynchronizer synchornizer;

    public void setSynchronizer(TBeanSynchronizer synchronizer)
    \{
        this.synchronizer = synchronizer;
    \}

    public void setUp() 
        throws Exception
    \{
        this.server = new Server();
    \}

    public void run()
        throws Exception
    \{
        this.server.start();
        this.synchronizer.sync( "server.started" );
        this.synchronizer.sync( "clients.done" );
        this.server.stop();
    \}

    public void tearDown()
        throws Exception
    \{
        this.server = null;
    \}
\}
\end{codelisting}
\caption{Example usage of \indexclass{TBeanSynchronizer} within
an implementation of \indexclass{SynchronizableTBean}.}
\label{example.synchronizable.tbean}
\end{figure}

The SysUnit framework provides an abstract base class to make
writing synchronizable \class{TBeans} easier.  The
\indexclass{AbstractSynchronizableTBean}
class provides for the management of the
\indexclass{TBeanSynchronizer}
and also exposes a
\indexmethod{AbstractSynchronizableTBean}{sync(...)}
method that works the same as the one provided to thread methods.
Additionally, it provides no-op implementations of all
other methods except for \indexclass{TBean}{run()}.
Figure \vref{example.abstract.synchronizable.tbean} reimplements
the same example as \vref{example.synchronizable.tbean} using
the \class{AbstractSynchronizableTBean} base class.

\begin{figure}
\begin{codelisting}
import org.sysunit.AbstractSynchronizableTBean;

public class ServerTBean
    extends AbstractSynchronizableTBean
\{
    public void run()
        throws Exception
    \{
        Server server = new Server();
        server.start();
        sync( "server.started" );
        sync( "clients.done" );
        server.stop();
    \}
\}
\end{codelisting}
\caption{Example usage of \indexclass{AbstractSynchronizableTBean}
utility base class.}
\label{example.abstract.synchronizable.tbean}
\end{figure}


\section{Details of Synchronization}

