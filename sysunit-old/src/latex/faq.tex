\newcommand{\faq}[1]{\section{\textsf{\normalsize{#1}}}}

\chapter{Frequently Asked Questions}

\faq{Why does JUnit complain that no tests were found?}

It is required that your \class{SystemTestCase} provide a
\emph{public}, \emph{static}, \indexmethod{SystemTestCase}{suite()}
method.  \class{SystemTestCase} provides a
\indexmethod{SystemTestCase}{suite(...)}
method to assist in the creation of the test suite.

\begin{codelisting}
import org.sysunit.SystemTestCase;
import junit.framework.TestSuite;

public class TestTheServer
    extends SystemTestCase
\{
    public static TestSuite suite()
    \{
        return suite( TestTheServer.class );
    \}
\}
\end{codelisting}

\faq{Why don't my \method{setUp()} and \method{tearDown()}
methods get called?}

In \class{SystemTestCase}, since each logical thread of execution
is fairly independent, common \method{setUp()} and \method{tearDown()}
methods are not supported.  SysUnit does not call these methods
even if they exist on your test case.

\faq{When would I use a \indexclass{TBean} instead of
a \emph{thread method}?}

\class{TBean}-based tests allow easy assembly and re-assembly
of common components to generate new and exciting tests.  For
simple tests, the creation of additional \class{TBean} classes
may be a burden, in which case \emph{thread methods} may 
be preferred.  Additionally, \class{TBean} instances \emph{do}
support \indexmethod{TBean}{setUp()} and
\indexmethod{TBean}{tearDown()} methods.

\faq{How do I adjust the watchdog timer without recompilation?}

Set the system property \texttt{org.sysunit.timeout.multiplier}
to any floating-point number to increase or decrease the watchdog
timer.
